\chapter*{Abstract}
\addcontentsline{toc}{chapter}{Abstract}
\vspace{-1em}
Network tomography is a technique used to make node-level inferences from end-to-end measurements of a network. It allows for an administrator to estimate router performance metrics without the costly overheads of traditional network monitoring approaches. We have adapted previous implementations of network tomography to estimate the queue buffer lengths of routers in stochastically routing networks, developing a technique we refer to as packet delay average (PDA) tomography. In this technique, we use the average delay of packets traversing a router as a proxy for the buffer queue length of the router. These buffer queue length estimates are subsequently used to infer the presence of \textit{nefarious routers}, which probabilistically delay packets. This probabilistic delaying represents compromised routers snooping network traffic via deep packet inspection or ex-filtrating encrypted data.\par
We have produced Python scripts for: calculation of an optimal selection of paths over the network to probe, optimal allocation of probes over these paths, and a network simulation tool using the iGraph package. We developed two classifiers (with and without access to network log files) to identify these nefarious routers in real-world ISP networks using PDA tomography. Both classifiers performed better than a random classification over all of the networks that we evaluated. An allocation of observable probe packets between paths, proportional to how many routers each was using to compute, improved classier sensitivity and specificity in two of the three networks we evaluated. Additionally, this proportional allocation increased the lower bound on the accuracy of network tomography in all tested networks.