%      $Id: macros.tex 506 2009-10-05 16:57:07Z daniel $   

% 
% Abbreviations
% 
\newcommand{\eg}{e.g., }
\newcommand{\ie}{i.e., }

\newcommand{\otoprule}{\midrule[\heavyrulewidth]}
\newcommand{\pdv}{Packet Delay Variation (PDV) }
\newcommand{\ur}{Uncontrollable routing (UR) }
\newcommand{\cfr}{Controllable cycle-free routing (CFR) }
\newcommand{\cbr}{Controllable cycle-based routing (CBR) }
\newcommand{\acr}{Arbitrarily controllable routing (ACR) }

\newcommand{\nurserytype}[1]{{\fontfamily{cmss}\selectfont \textsl{#1}}}
\newcommand{\alloc}{\nurserytype{allocate}\xspace}
\newcommand{\collect}{\nurserytype{collect}\xspace}
\newcommand{\redirect}{\nurserytype{redirect}\xspace}
\newcommand{\bmtype}[1]{{\textsf{#1}}}
\newcommand{\doi}[1]{\href{http://dx.doi.org/#1}{\nolinkurl{doi:#1}}}
% /Abbreviations

% 
% Math operators
% 
\DeclareMathOperator*{\argmax}{argmax}
\DeclareMathOperator*{\argmin}{argmin}
\DeclareRobustCommand{\bbone}{\text{\usefont{U}{bbold}{m}{n}1}}
\DeclareMathOperator{\EX}{\mathbb{E}}
\DeclareMathOperator{\VAR}{\hat{\mathbb{V}}}
\newcommand{\probP}{\text{I\kern-0.15em P}} 
\newtheorem{theorem}{Theorem}[section]
\newtheorem{corollary}{Corollary}[theorem]
\newtheorem{lemma}[theorem]{Lemma}
% /Math operators

% 
% Custom colours
% 
\definecolor{tableheadcolor}{rgb}{0.8,0.8,1.0}
%\definecolor{tablealtcolor}{rgb}{0.9,0.9,1.0}
\definecolor{tablealtcolor}{rgb}{0.9,0.9,0.95}
\definecolor{todocolor}{rgb}{0.8,0.8,1.0}
\definecolor{fixcolor}{rgb}{1,0.8,0.8}
\definecolor{commentcolor}{rgb}{0.8,1.0,0.8}
% /Custom colours

%
% Margin notes
%
\newcommand{\ltodo}[1]{\reversemarginpar\todo[color=todocolor]{#1}}
\newcommand{\rtodo}[1]{\normalmarginpar\todo[color=todocolor]{#1}}
\newcommand{\itodo}[1]{\todo[inline]{#1}}

\newcommand{\lfix}[1]{\reversemarginpar\todo[color=fixcolor]{#1}}
\newcommand{\rfix}[1]{\normalmarginpar\todo[color=fixcolor]{#1}}
\newcommand{\ifix}[1]{\todo[inline,color=fixcolor]{#1}}

\newcommand{\lcomment}[1]{\reversemarginpar\todo[color=commentcolor]{#1}}
\newcommand{\rcomment}[1]{\normalmarginpar\todo[color=commentcolor]{#1}}
\newcommand{\icomment}[1]{\todo[inline,color=commentcolor]{#1}}

% 
% Misc
% 

\newsavebox{\tempbox}

\newcommand{\textbox}[1]% #1 = text
{\savebox{\tempbox}{#1}% get width
 \ifdim\wd\tempbox<4cm\relax
   \makebox[4cm]{\usebox{\tempbox}}%
 \else
   \parbox{4cm}{\raggedright #1}%
 \fi}

\renewcommand\theadfont{\normalsize\bfseries}
\lstloadlanguages{python}
\DeclareGraphicsRule{*}{pdf}{*}{}
\newcommand{\ignore}[1]{}
\newcommand{\mccenter}[1]{\multicolumn{1}{c|}{#1}}

\newcommand{\placeholderfigure}[2]{
\begin{figure}[ht!]
  \begin{center}
    \resizebox{\textwidth-2cm}{0.7\textwidth-1.4cm}{todo}
  \end{center}
  \caption{#2}#1
\end{figure}}

\newcommand{\rowvect}[1]{% inline row vector
  \begin{bmatrix}#1\end{bmatrix}%
}

\long\def\sfootnote[#1]#2{\begingroup%
\def\thefootnote{\fnsymbol{footnote}}\footnote[#1]{#2}\endgroup}

\newcommand\MyBox[2]{
  \fbox{\lower0.75cm
    \vbox to 1.7cm{\vfil
      \hbox to 1.7cm{\hfil\parbox{1.4cm}{#1\\#2}\hfil}
      \vfil}%
  }%
}
% /misc

%
% crossreferencing footnotes
%
%\newcommand{\fnref}[1]{~(\ref{#1})}
%\newcommand{\onecolparbox}{3.1in}


%%
%% Change the sections etc.
%%
%\makeatletter
%\parskip=0pt
%\renewcommand\section{\@startsection{section}{1}{\z@}%
%                                   {-2.5ex}% beforeskip
%%                                   {1ex}% afterskip
%                                   {\large \bfseries \raggedright}}
% \renewcommand\subsection{\@startsection{subsection}{2}{\z@}%
%                                     {-2ex\@plus -1ex \@minus -.2ex}%
%                                      {.5ex \@plus .2ex}%
%                                      {\normalsize \bfseries \raggedright}}
% \renewcommand\subsubsection{\@startsection{subsubsection}{3}{\z@}%
%                                      {-2ex\@plus -1ex \@minus -.2ex}%
%                                      {1ex \@plus .2ex}%
%                                      {\normalfont\fontsize{11pt}{12pt}\selectfont\itshape}}
%\renewcommand{\thesubsubsection}{\thesubsection.\arabic{subsubsection}}

%\renewcommand\paragraph{\@startsection{paragraph}{4}{\z@}% 
%  {.5em}%
%  {-1em}%
%  {\normalfont\normalsize\bfseries\parskip=0pt}}
%\setlength\partopsep{0\p@}
%\setlength\parskip{0\p@ \@plus \p@}

%\makeatother
%\parindent=9pt

%%% Local Variables: 
%%% mode: latex
%%% TeX-master: "doa"
%%% End: